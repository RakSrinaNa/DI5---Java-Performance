\documentclass{report}
\usepackage{MCC}

\def\footauthor{Thomas COUCHOUD \& Victor COLEAU}
\title{Concurrent collections - TP3}
\author{Thomas COUCHOUD\\\texttt{thomas.couchoud@etu.univ-tours.fr}\\Victor COLEAU\\\texttt{victor.coleau@etu.univ-tours.fr}}

\rowcolors{1}{white}{white}
\begin{document}
	\mccTitle
	
	\chapter{Qu'est-ce que c'est ?}
		\section{Collection}
			En Java Collection désigne \href{https://docs.oracle.com/javase/8/docs/api/java/util/Collection.html}{une interface}.
			Cette dernière a pour but de désigner un groupe d'objets appelés éléments.
			Cette définition est très générique, ce qui laisse les implémentations variées.
			Certaines peuvent être ordonnées, d'autres non ; certaines avec duplicatas, d'autres non.
			
			Dans le JDK on ne trouve pas d'implémentation directe de cette interface, seulement des implémentations d'interfaces plus spécifiques telles que List ou Set.
			
			\img{diag.png}{}{}
			
			
			Collection définie notamment les méthodes add(element), contains(element) et remove(element).
			Parlons donc rapidement des 3 grandes extensions de Collection: List, Set, Queue.
			\begin{easylist}[itemize]
				@ List définie une collection d'objets ordonnée.
				On peut avoir des éléments dupliqués et il y a du sens de parler de l'élément à l'indice $k$.
				De ce fait la méthode remove(index) est aussi définie dans ce cas.
			
				\textbf{Exemple}: $\left[1,2,3,4,3,5,4\right]$
				
				@ Queue désigne une queue. Cette dernière n'est pas très différente d'une list: une collection d'éléments ordonnés.
				Cependant on ne peux que manipuler les objets se trouvant uniquement aux extrémités (on ajoute d'un coté, et on retire d'un autre).
				Parler de l'élément à l'indice $k$ n'a donc aucun sens, on le verra quand il est en tête de queue.
				
				On retrouve les fonctions offer(element), poll et peek.
				
				@ Set définie une collection d'objets n'ayant pas d'ordre et pas de duplicata.
			\end{easylist}
			
		\section{Concurrence}
			\lstinputlisting[caption=Main.java, language=JAVA]{a.java}
			
			Si on lance le code ci-dessus, nous allons obtenir une exception java.util.ConcurrentModificationException.
			En effet, nous avons deux threads qui tentent d'accéder à la même liste en même temps (le while et notre add ligne 11).
			
			Afin de pouvoir réaliser ces opérations, deux possibilités s'offrent à nous : soit on gère cela dans notre code pour être sûr qu'une seule opération est réalisée à la fois, soit on utilise des implémentations de collections qui gère la concurrence.
			
			Deux façons de procéder reposent sur Collections.synchronizedXxx() et le package java.util.concurrent.
			
	\chapter{Comment ça marche ?}

		Le but de ces solutions est de créer des objets ThreadSafe. C'est à dire qu'ils peuvent être manipulés par plusieurs Threads parallèles sans provoquer d'erreur.

		Il est à noter que certaines classes spécifiques de Java.util sont nativement ThreadSafe telles que Vector et Stack.

		\section{Collections.synchronizedXxx()}

			Ces méthodes static prennent toutes en paramètres un objet de leur type (un Set pour Collections.synchronizedSet, une List pour Collections.synchronizedList, etc.) et renvoient un objet "synchronisé" de même type.

			Toute interaction avec la Collection devra s'effectuer au travers de ce nouvel objet.
			Il est donc recommandé de supprimer la référence à l'objet "non-synchronisé".
			De plus, s'il l'on souhaite itérer sur la dite Collection, il faut le faire dans un block \textit{synchronized} avec comme verrou la Collection.

			Cette solution, bien que simple à mettre en place, comprend un défaut majeur.
			En effet, un objet "synchronisé" est un objet dont toutes les méthodes sont synchronisées.
			Cela signifit qu'à chaque appelle d'une de ses méthodes, tout l'objet est verrouillé, même dans le cas de méthodes non-problèmatiques telles que Get (dans le cas d'une List).

			\img{lockC.png}{}{scale=0.8}

			De ce fait, l'accès à l'objet est très ralenti car les processus de verrouillage et déverrouillage sont couteux en temps, ce qui provoque un effet de congestion.		

		\section{java.util.concurrent}

			Ce package contient des classes crées spécifiquement pour gérer la concurrence et les problèmes qu'elle soulève.
			On y trouve par exemple ConcurrentHashMap, CopyOnWriteArrayList.

			L'avantage de cette solution par rapport à la précédente est que celle-ci ne bloque pas tout l'objet mais utilise la technique du Lock Striping.

			\img{lockB.png}{}{scale=0.8}

			Cette technique consiste à verrouiller une portion d'une structure de données où chaque lock verouille un ensemble d'objets indépendants (appelé Bucket).
			Dans le cas d'un ConcurrentHashMap on dispose par défaut de 16 Buckets donc de 16 locks différents.
			Grâce à cela, jusqu'à environ 16 opérations normalement concurrentes peuvent être ici réalisées en parallèle.
			

	\chapter{Quels sont les impacts sur les performances ?}
\end{document}
