\documentclass{report}
\usepackage{MCC}

\def\footauthor{Thomas COUCHOUD \& Victor COLEAU}
\title{Java Performance - TP4}
\author{Thomas COUCHOUD\\\texttt{thomas.couchoud@etu.univ-tours.fr}\\Victor COLEAU\\\texttt{victor.coleau@etu.univ-tours.fr}}

\begin{document}
	\mccTitle
	
	\chapter{Introduction}
	
	\chapter{Exercices}
	\section{Exercice 1}
		On remarque que beaucoup de types utilisés des classes (Float, Integer, ...), il serait donc peut-être plus efficace d'utiliser des types primitifs directement afin d'éviter d'inutiles opérations d'unboxing/boxing.
		
		De plus nous avons changé les "k = k + 1" en "k++".
		
		Après des mini benchmarks nous obtenons la sortie suivante:
		\lstinputlisting[caption=benchmark.txt]{jmh/exo1_Bench.txt}
		
		Avec nos améliorations on peut constater que le score est multiplié par environ 5.
		Cela est probablement du aux changement des types plus qu'à la transformation des incrémentations.

	\section{Exercice 2}
		La première remarque est que dans le calcul d'une valeur de fibonacci, le calcul des deux précédentes est nécessaire.
		Or dans l'implémentation donné fibonacci(i-1) et fibonacci(i-2) sont indépendantes alors que fibonacci(i-1) utilise et donc recalcule lui-même fibonacci(i-2).
		Ce dernier calcul est donc effectué au moins 2 fois, ce qui est inutule et peut être très long sur des grandes valeurs.
		
		Nous allons donc mettre en place un système de cache des valeurs au travers d'une map (fibonaciA).
		
		De plus, étant donné que l'on demande la valeur de fibonacci(43), nous pouvons aussi renvoyer la valeur directement en l'ayant calculée auparavant (fibonacciB).
		
		\lstinputlisting[caption=benchmark.txt]{jmh/exo2_Bench2.txt}
		
		L'amélioration est ici flagrante avec le cache, et encore plus avec la valeur renvoyée directement.
		
	
	\section{Exercice 3}
		Dans les deux exercices suivants, nous n'avons pas pu comparer avec les méthodes de base car ces dernières font crasher la VM.
		Nous pouvons cependant déduire les potentielles améliorations grâce au temps d'exécutions des tests unitaires.
		En effet les méthodes de bases durent environ 1m30 alors que les versions améliorées sont aux alentours de 30s (pour 3a) et 1m (pour 3b).
	
		\subsection{Exercice 3a}
			Dans une première méthode "A" nous avons:
			\begin{easylist}
				@ Retiré le synchronized de la méthode. En effet on a déjà un mutex à l'intérieur de la fonction et il n'est donc pas nécessaire d'en avoir deux. De plus le synchronized sur la méthode bloque toutes les autres méthodes synchronized ce qui n'est pas optimal.
				@ Changé les Integer en int afin d'éviter les boxing/unboxing inutiles.
				@ Ajout direct du future dans la liste au lieux de la création d'un objet intermédiaire.
			\end{easylist}
			
			Une deuxième méthode "B" reprend les modifications de la méthode "A" mais utilise cette fois-ci un AtomicInteger afin de remplacer le mutex dans la fonction d'incrémentation.
			
			Les résultats sont les suivants:
			\lstinputlisting[caption=benchmark.txt]{jmh/exo3a_Bench.txt}
		
		\subsection{Exercice 3b}
			Nous avons ici repris les mêmes idées que dans l'exercice 3a cependant nous avons du changer le incrementAndGet du AtomicInteger en accumulateAndGet afin d'y ajouter le calcul du modulo.
			
			Les résultats sont les suivants:
			\lstinputlisting[caption=benchmark.txt]{jmh/exo3b_Bench.txt}
			
			

	\section{Exercice 4}
		La première chose que nous remarquons est que l'on fait une boucle for pour ajouter tous les éléments d'un tableau.
		
		La première méthode consistant à utiliser le .addAll de la liste (exercice4A).
		Cependant cela nous a forcé à convertir pleins de types ce qui ne va probablement pas être optimal.
		
		Une deuxième méthode a été d'utiliser directement les streams avec un flatMap (exercice4C), cependant encore une fois cela implique de faire des conversions de type.
		
		Notre dernière méthode consiste à d'abord calculer la taille du tableau final, puis le remplir directement (exercice4B).
		Cela évite d'ajouter une fois dans une liste, puis d'ajouter dans un tableau.
		
		\lstinputlisting[caption=benchmark.txt]{jmh/exo4_Bench.txt}
		
		Le A est environ deux fois plus lent que l'original.
		Ce qui prouve que notre conversion de types n'est pas efficace du tout et qu'il vaut mieux travailler directement des bytes.
		
		Le C est environ 1,5 fois plus rapide.
		En effet comparé à la méthode A, on change toujours le type mais on s'affranchit de la liste intermédiaire et construisons directement le tableau final.
		
		Enfin la méthode B est la meilleure en étant 2 à 3 fois plus rapide que l'originale.
		Cela est du au fait que l'on créé directement le tableau final mais cette fois sans aucune conversion de type.
	
	\section{Exercice 5}
		Ici la réfléxivité est utilisée sur un objet dont on connait déjà le type et qui contient déjà la méthode à appeler.
		Tout cet enchainement est inutile car nous pouvons appeler le .getName() directement sur notre objet.
		Cela va nous permettre de s'affranchir de la recherche de la méthode et invokation de cette dernière.
		
		\lstinputlisting[caption=benchmark.txt]{jmh/exo5_Bench.txt}
		
		Encore une fois on remarque que ces changements ont grandement augmenté les performances.
	
	\chapter{Conclusion}
\end{document}
