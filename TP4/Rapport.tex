\documentclass{report}
\usepackage{MCC}

\def\footauthor{Thomas COUCHOUD \& Victor COLEAU}
\title{Java Performance - TP4}
\author{Thomas COUCHOUD\\\texttt{thomas.couchoud@etu.univ-tours.fr}\\Victor COLEAU\\\texttt{victor.coleau@etu.univ-tours.fr}}

\begin{document}
	\mccTitle
	
	\chapter{Introduction}
	
	\chapter{Exercice 1}
		On remarque que beaucoup de types utilisés des classes (Float, Integer, ...), il serait donc peut-être plus efficace d'utiliser des types primitifs directement afin d'éviter d'inutiles opérations d'unboxing/boxing.
		
		De plus nous avons changé les "k = k + 1" en "k++".
		
		Après des mini benchmarks nous obtenons la sortie suivante:
		\lstinputlisting[caption=benchmark.txt]{jmh/exo1_Bench.txt}
		
		Avec nos améliorations on peut constater que le score est multiplié par environ 5.
		Cela est probablement du aux changement des types plus qu'à la transformation des incrémentations.

	\chapter{Exercice 2}
		La première remarque est que dans le calcul d'une valeur de fibonacci, le calcul des deux précédentes est nécessaire.
		Or dans l'implémentation donné fibonacci(i-1) et fibonacci(i-2) sont indépendantes alors que fibonacci(i-1) utilise et donc recalcule lui-même fibonacci(i-2).
		Ce dernier calcul est donc effectué au moins 2 fois, ce qui est inutule et peut être très long sur des grandes valeurs.
		
		Nous allons donc mettre en place un système de cache des valeurs au travers d'une map.
		
		\lstinputlisting[caption=benchmark.txt]{jmh/exo2_Bench.txt}
		
		L'amélioration est ici flagrante.
		
	
	\chapter{Exercice 3}
		\section{Exercice 3a}
		
		\section{Exercice 3b}

	\chapter{Exercice 4}
	
	\chapter{Exercice 5}
	
	\chapter{Conclusion}
\end{document}
