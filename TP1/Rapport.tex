\documentclass{report}
\usepackage{MCC}

\def\footauthor{Thomas COUCHOUD \& Victor COLEAU}
\title{Java Performance - TP1}
\author{Thomas COUCHOUD\\\texttt{thomas.couchoud@etu.univ-tours.fr}\\Victor COLEAU\\\texttt{victor.coleau@etu.univ-tours.fr}}

\rowcolors{1}{white}{white}
\begin{document}
	\mccTitle
	
	\chapter{Test des coutils}
		\section{vmstat}
			vmstat est un outil en ligne de commande permettant d'obtenir différentes informations sur la VM Java:
			\begin{easylist}[itemize]
				@ Nombre de processus:
				@@ Processus en cours d'exécution
				@@ Processus dormants qu'on ne peut interrompre
				@ Mémoire:
				@@ Mémoire virtuelle utilisée
				@@ Mémoire libre
				@@ Mémoire buffer
				@@ Mémoire cache
				@@ Mémoire inactive
				@@ Mémoire active
				@ Swap
				@@ Entrée
				@@ Sortie
				@ IO
				@@ Lecture
				@@ Ecriture
				@ Système
				@@ Nombre d'interruptions par seconde
				@@ Nombre de changement de contexte
				@ CPU
				@@ Temps en non-kernel
				@@ Temps en kernel
				@@ Temps en attente
				@@ Temps en attente d'IO
				@@ Temps pour la VM
			\end{easylist}
			
			\img{vmstat1.png}{Sortie vmstat au lancement d'Eclipse}{0.5}
			
			En observant notamment les colonnes mémoire libre;tampon;cache, io bi, cpu us on peut remarquer que:
			\begin{easylist}[itemize]
				@ La mémoire libre diminue tandis que la mémoire cache augmente
				@ La lecture des fichiers explose
				@ Une légère augmentation de l'écriture de fichiers de temps en temps
				@ Le temps CPU non kernel augmente tendis que le temps CPU en attente diminue.	
			\end{easylist}

			Vmstat nous permet donc d'obtenir des informations globales sur l'état du système et non de la VM Java.
			Il n'est pas possible de faire une analyse poussée des besoins en resources d'un programme en particulier.		
				
		\section{iostat}
			Iostat permet d'obtenir des information générales sur les IOs (par périphérique) et CPU de tout le système.
			Cela peut être répété $n$ fois avec un interval donné.
			
			\img{iostat1.png}{iostat avec Eclipse ouvert}{0.5}
			
			De manière similaire à vmstat, cet outil est assez peu précis concernant Java étant donné que les statistiques obtenues concernent tout le système.
			Des interférences d'autres programmes peuvent altérer notre analyse.
			
		\section{nicstat}
			Cet outil est similaire à iostat à la différence qu'il analyse les données transitant par les cartes réseau.
			De ce fait il ne serait utile que le cas d'un programme Java utilisant la connectivité réseau.
			Il faut cependant faire attention à ce que d'autres programmes ne se servent pas du réseau en même temps.
			
			\img{nicstat1.png}{Evolution du réseau lors du rafraichissement des packages disponibles dans Eclipse}{0.5}
			
			On peut imaginer que ces 3 derniers outils très simplistes peuvent cependant être utilisés dans dans scripts afin de faire du monitoring.
			
		\section{JCMD}
			\cbo{jcmd} permet d'obtenir les différents processus Java lancés.
			
			On peut par la suite lancer \cbo{jcmd <PID> <command>} pour effectuer des commandes sur la JVM.
			On retrouve notamment la commande help pour obtenir la liste des commandes disponibles.
			
			La commande \cbo{Thread.print} affiche tous les threads lancés ainsi que leur status et pile d'appel.
			
			La commande \cbo{GC.run} permet de lancer le GC manuellement.
			
			De manière générale JCMD se connecte à la VM pour exécuter des commandes pour obtenir des informations sur la VM mais aussi pour exécuter des actions sur celle-ci.
			En comparaison avec les précédents, cet outil, bien que plus puissant n'est que peu intégrable à des scripts dû aux différentes commandes à appeler dans la nouvelle console.
			
			Un avantage non négligeable de JCMD est que, étant connecté directement à la JVM, il n'est pas influencé par des processus externes.
			
		\section{JConsole}
			On retrouve des informations que nous avons déjà pu voir auparavant (Mémoire, Threads, etc.) mais sous forme graphique.
			
			Grâce à cette présentation, il est possible d'analyser l'évolution de la JVM au cours du temps ainsi que de pouvoir corréler différentes métriques d'un simple coup d'\oe il.
			De plus, ce type d'interface offre un gain de temps considérable dû à l'aisance de navigation entre les différentes informations.
			Pour la même raison, un utilisateur débutant pourra trouver les informations qu'il cherche sans avoir à retenir toutes les commandes JCMD.
						
			\begin{figure}[H]
				\begin{minipage}{0.49\textwidth}
					\img{jconsole1.png}{Eclipse en fond (vue générale)}{0.25}
				\end{minipage}
				\begin{minipage}{0.49\textwidth}
					\img{jconsole2.png}{Eclipse en fond (Mémoire heap)}{0.25}
				\end{minipage}
			\end{figure}
			
			Cependant nous devons nous connecter à une application déjà lancée.
			Il est donc difficile de récupérer les statistiques au lancement d'une application.
			
		\section{JHat}
			JHat permet d'analyser les fichier hprof de la JVM.
			Ceux-ci contiennent des informations sur le CPU et la mémoire heap.
			
			Nous pouvons un de ces fichiers à la volée grâce à \cbo{jmap -dump:file=<file> <pid>}.
			Ils sont aussi potentiellement créés lorsque la JVM crash.
			
			Il nous suffit ensuite de l'ouvrir avec \cbo{jhat <file>}.
			Dès que le fichier est analysé, un serveur HTTP est crée.
			Nous retrouvons sur ce dernier les informations du fichier sous forme visuelle.
			
			Ce type de fichier trouve toute son utilité après un crash de la JVM car il permet de remonter aux derniers instants de vie de la VM.
			Cela peut donc permettre de trouver la cause du crash.
			
			\img{jhat1.png}{Toutes les classes}{0.4}
			\img{jhat2.png}{Détails d'une classe}{0.4}
			
			
		\section{JStack}
			Affiche la liste des threads de la VM, similaire à \cbo{Thread.print} de JCMD.
			
			Cela n'a que peu d'intérêt lorsque l'on maitrise JCMD mais peut permettre à un débutant ou à un script d'accéder aux informations sur les threads.
			
		\section{JVisualVM}
			Outil similaire à JConsole affichant des graphes des différentes métriques de la VM.
			
			Il est possible d'ajouter des plugins ce qui offre une multitude de possibilités selon nos besoins.
			On peut notamment citer le plugin GC permettant d'avoir des informations sur le garbage collector.
			
		\section{Java Mission Control}
			Encore une interface similaire aux autres, affichant de nombreux graphiques et tableaux sur l'état de la VM et permet aussi l'ajout de plugins pour étendre les fonctionnalités.
			
			De plus il offre la possibilité d'enregistrer sur une période de temps donnée des métriques choisies.
			Cette fonctionnalité de cibler les actions que l'on veut étudier en enregistrant que l'évolution du processus lors de ces actions.
			Additionnellement on peut personnaliser les métriques étudies en ne sélectionnant que celles que l'on souhaite ainsi que leur niveau de détail.
			
			la possibilité de sauvegarder ces résultats permet de les analyser plus tard voir même de les archiver pour comparaisons ultérieures.
			
			\img{jmc2.png}{Page GC de Java Mission Control}{0.3}
			
		\section{Conclusion}
			Parmi tous les outils étudiés, on remarque deux catégories:
			\begin{easylist}[itemize]
				@ Les outils console:
				@@ Outils analysant le système: Ils permettent une analyse générale de tout l'état du système.
				Cela peut permettre de détecter des comportement anormaux mais peut être fortement sujet à des perturbations externes. 
				Il faut donc les utiliser avec prudence et ne pas tirer de conclusions trop rapidement.
				@@ Outils analysant la JVM: Ceux-ci sont plus précis que les derniers car ils ciblent directement la JVM et ne sont donc pas perturbés par les processus externes.
				Certains d'entre eux peuvent utilisés de manière automatisée dans des scripts afin de faire du monitoring.
				@ Les outils graphiques: Ils sont bien plus simple d'utilisation pour un utilisateur humain mais sont difficiles (voir impossibles) à automatiser.
				De plus ils présentent les informations de manière graphique au cours du temps ce qui permet une analyse rapide pour ensuite cibler plus précisément le problème.
				Enfin certains d'entre eux proposent l'ajout de plugins afin d'étendre leur fonctionnalités selon les besoins.
			\end{easylist}

			
\end{document}